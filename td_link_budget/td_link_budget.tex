\documentclass{article}
\usepackage[utf8]{inputenc}
\usepackage{graphicx}
\usepackage{enumerate}
\usepackage{amsmath}
\usepackage{parskip}
\usepackage{siunitx}

\usepackage{geometry}
 \geometry{
 a4paper,
 total={170mm,257mm},
 left=20mm,
 top=20mm
 }

\title{TD: 5G NR mmW Link Budget}
\author{Markus Säynevirta}
\date{June 2022}

\begin{document}

\thispagestyle{plain}

\large
\textbf{RIO207 - Ingénierie radio}

\large
TD: 5G NR mmW Link Budget\\
\textit{Markus Säynevirta}
\vspace{0.5cm}

\section{Transmission}
\subsection{EIRP and transit diversity gain (question 1)}

EIRP of the transmitter can be calculated with the formula 

\begin{align*}
     \mathrm{EIRP(dBm)} = P_{tx}\mathrm{(dBm)} - L\mathrm{(dB)} + G\mathrm{(dBi)}
\end{align*}

In relation to the given parameters, gains comprise only of the transmit diversity gain, as antenna element gain is defined as \(G_t = 0\ \mathrm{dBi}\). Transmit diversity typically yields an additional gain of \(G =3\ \mathrm{dB}\), thus we arrive at \(\mathrm{EIRP} = 26\ \mathrm{dB}\).

\section{Reception}
\subsection{Target SINR (question 2)}

Target SINR can be formulated from the Shannon theorem by rearranging the formula to solve for the signal-to-noise ratio SNR. For the sake of simplicity, interference is considered as noise, and thus SINR is assimilated as SNR.

\begin{gather*}
     \mathrm{C} = B\log_2 (1+\mathrm{SNR}) \\
     \Leftrightarrow \mathrm{SNR} = 2^{\frac{C}{B}} - 1
\end{gather*}

Given the parameters of \(B = \SI{20}{\mega bps}\) and \(C = \SI{100}{\mega\hertz}\), we get a \(\mathrm{SINR_{target}} = 0.148\) or \(\SI{-8.276}{\deci\bel}\).

\subsection{Noise power at receiver (question 3)}

We can compute noise power with the formula 

\begin{align*}
     \mathrm{N} &= N_0 W \cdot \mathrm{NF}
\end{align*}

Noise figure NF was given as \(\SI{3}{\deci\bel}\) yielding an overall noise power of \(-174 + 10 \log_{10} (W) = \SI{-94}{\deci\bel m}\).

\subsection{Receiver gains (question 4)}

The BS under consideration has 128 antenna elements (\#AE) and 64 TXRU (\#TXRU) while it uses 2 polarizations (\#AP) for receive diversity. The antenna element gain is given as \(G_{ae} =  \SI{8}{\deci\bel i}\).

Receiver array gain can be calculated as

\begin{align*}
    G_{array} = 10 \log_{10} (\mathrm{\#AE}/\mathrm{\#AP}) = \SI{18.061}{\deci\bel}
\end{align*}

Receiver antenna gain is equal to the gain of an individual antenna element or \(G_{\mathrm{antenna}} = \SI{8}{\deci\bel i}\). Diversity gain \(G_{\mathrm{diversity}} = \SI{3}{\deci\bel}\), as the two polarizations are used in receive diversity.

\subsection{Sensitivity of the receiver (question 5)}

HARQ gain can be calculated from the amount of retransmissions. Given 4 HARQ retransmissions, we get a HARQ gain of \(G_{\mathrm{HARQ}} = 10 \log_{10} (N) = \SI{6.02}{\deci\bel}\). Additionally, scheduling yields typically an additional gain of  \(\SI{3}{\deci\bel}\). With this information and considering the results worked out in the previous subsections, the sensitivity of the receiver can be calculated as

\begin{align*}
    S_{receiver} = \mathrm{SINR_{target}} + N + \mathrm{NF} - \sum G = \SI{-137.35}{\deci\bel m}
\end{align*}

\section{Margins}
\subsection{Shadowing margin (question 6)}
Shadowing margin can be calculated with Jakes' formula:

\begin{align*}
      K_s= \sigma Q^{-1}(P_{out})
\end{align*}

Given a shadowing standard deviation of \(\sigma = 6\) and assuming a constraint of \(0.99\), we get a shadowing margin of \(\SI{13.98}{\deci\bel}\).

\subsection{Hand and body loss (question 7)}
\subsection{Indoor penetration loss (question 8)}
3GPP technical report 38.901 defines a O2I penetration loss model:

\begin{align*}
    \mathrm{PL} = \mathrm{PL_b} + \mathrm{PL_{tw}} + \mathrm{PL_{in}} + N(0, \sigma^2_P)
\end{align*}

The different path losses can be calculated from the following equations assuming the low-loss model, while the standard deviation is defined as \(\SI{4.4}{\deci\bel}\) in the model. \(P_b\) is the basic outdoor path loss. We will use the PL’\_UMa-NLOS loss model similar to the cell radius modelling in section \ref{cell_r}.

\begin{gather*}
    \mathrm{PL_{tw}} = 5-10\log_{10} (0.3 \cdot 10^{\frac{-L_{\mathrm{glass}}}{10}} + 0.7 \cdot 10^{\frac{-L_{\mathrm{concrete}}}{10}}) \\ \\
    \mathrm{PL_{in}} = 0.5 \cdot d_{2D-in}
\end{gather*}

Penetration losses in different materials can be calculated from the table 7.4.3-1, giving values \(L_{\mathrm{glass}} = \SI{7.6}{\decibel}\) and \(L_{\mathrm{concrete}} = \SI{117}{\decibel}\) at a frequency of 28\ GHz.

Solving the penetration component of the path loss equation and summing it with the indoor loss and standard deviation components gives us an overall margin figure of

\begin{gather*}
    \mathrm{PL_{tw} + \mathrm{PL_{in} + \sigma^2_P = \SI{22.72}{\deci\bel}}}
\end{gather*}

\section{Cell radius}
\label{cell_r}

\section{Deployment scenario}
Paris 13th arrondissement

\end{document}
