\documentclass{article}
\usepackage[utf8]{inputenc}
\usepackage{graphicx}
\usepackage{enumerate}
\usepackage{amsmath}
\usepackage{parskip}
\usepackage{siunitx}

\usepackage{geometry}
 \geometry{
 a4paper,
 total={170mm,257mm},
 left=20mm,
 top=20mm
 }

\title{TP: Cellular Network Planning on a 2D Map}
\author{Markus Säynevirta}
\date{June 2022}

\begin{document}

\thispagestyle{plain}

\large
\textbf{RIO207 - Ingénierie radio}

\large
TP: Cellular Network Planning on a 2D Map\\
\textit{Markus Säynevirta}
\vspace{0.5cm}

\section{Pre-dimensioning based on pilot coverage}
\subsection{Link budget (question 1)}

In- and outdoor sensitivities were computed for the pilot signal as part of a link budget spreadsheet included in the exercise zip archive. Table \ref{tab:link_budget} presents an excerpt containing the sensitivity values for both coverage scenarios as well as cell radii computed for these same scenarios.

\begin{table}[!htb]
    \centering
    \begin{tabular}{|l|l|l|}
    \hline
    \multicolumn{3}{|c|}{\textbf{Sensitivities}} \\ \hline
    Sensitivity, indoor {[}dBm{]}          & -92.92  &                                       \\ \hline
    Sensitivity, outdoor {[}dBm{]}         & -107.92 &                                       \\ \hline
    \multicolumn{3}{|c|}{\textbf{Cell range}} \\ \hline
    MAPL indoor {[}dB{]}                   & 140.43  & given value                           \\ \hline
    MAPL outdoor {[}dB{]}                  & 155.43  & given value                           \\ \hline
    Hata A                                 & 134.35  & given value                           \\ \hline
    Hata B                                 & 38.98   &                                       \\ \hline
    Hata C                                 & 0.44    &                                       \\ \hline
    Cell range, indoor {[}km{]}            & 1.47    &                                       \\ \hline
    Cell range, outdoor {[}km{]}           & 3.56    &                                       \\ \hline
    \end{tabular}
\caption{An excerpt from the link budget.}
\label{tab:link_budget}
\end{table}

\subsection{Site coverage and number (question 2)}

Table \ref{tab:cell_numbers} presents approximate numbers of cells in the two previously outlined deployment scenarios. Numbers have been computed by dividing the service area of \(\SI{100}{\kilo\metre\squared}\) with the area of the hexagonal cell calculated from the equation \(A_c = \frac{3 \sqrt{3} R^2}{8} \).

\begin{table}[!htb]
    \centering
    \begin{tabular}{|l|l|}
    \hline
    \multicolumn{2}{|c|}{\textbf{Number of cells, indoor}} \\ \hline
    Cell area {[}km²{]}                    & 1.40                                        \\ \hline
    Number of cells                        & 72.00                                       \\ \hline
    \multicolumn{2}{|c|}{\textbf{Number of cells, outdoor}} \\ \hline
    Cell area {[}km²{]}                    & 8.25                                        \\ \hline
    Number of cells                        & 13.00                                       \\ \hline
    \end{tabular}
    \caption{Number of cells in the different deployment scenarios.}
    \label{tab:cell_numbers}
\end{table}

\section{Cell planning based on coverage}
\subsection{Site planning proposal (question 3)}
\subsection{Cell loading (question 4)}
\section{Capacity study}
\subsection{Pre-dimensioning based on cell capacity}
\subsection{Cell planning based on capacity}
\end{document}
